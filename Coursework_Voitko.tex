\documentclass[oneside,final,12pt]{extreport}

\usepackage[T2A]{fontenc}
\usepackage[utf8]{inputenc}
\usepackage{vmargin}
\usepackage[english, russian]{babel}
%\usepackage{chemformula}
\usepackage[version=4]{mhchem}
\usepackage{indentfirst}

\begin{document}

% ============== Title page begin ==============

\centerline{БЕЛОРУССКИЙ  ГОСУДАРСТВЕННЫЙ УНИВЕРСИТЕТ}

\vfill
\vfill
\vfill
\large
\centerline{Войтко Глеб Георгиевич}
\vfill
\Large
\begin{center}
{\bf Наноструктурированные поверхности  диоксида титана для создания фотокаталитически активных гетероструктур}
\end{center}
\normalsize
\vfill
\centerline{Специальность 000.00.00 --- }
\centerline{Научно-исследовательская}
\vfill
\centerline{Курсовая работа}
\vfill
\vfill
\begin{flushright}
Научный руководитель:\\
д. Хорошко Л. С.
\end{flushright}
\vfill
\vfill
\centerline{Минск 2023}
\thispagestyle{empty}

% ============== Title page end ==============

% ============== Table of contents start ==============
\tableofcontents
% ============== Table of contents start ==============

% ============== Main content start ==============
\begin{chapter} {Введение}

Несомненно, возможность проведения таких сложных химических процессов, как инактивация бактерий или превращение \ce{CO2} в топливо, без использования высоких давлений или повышенных температур и с использованием света в качестве единственного источника энергии, интересна не только с инженерной точки зрения, но и с фундаментальной точки зрения. По этой причине в последние десятилетия активно проводятся исследования в области создания наноматериалов для фотокатализа. Область нанотехнологий вызвала большой интерес прежде всего потому, что в наномасштабе материалы обладают многочисленными новыми и врожденными свойствами. Эти свойства, зависящие от размера, включают новое поведение фазового перехода, особые термические и механические свойства, интересную поверхностную активность и реакционную способность (катализ), а также необычные оптические, электрические и магнитные характеристики [2]. Целью этой работы ставится исследовние методов получения и использования наноструктурированных поверхностей диоксида титана, рассмотрение его фотокаталитических свойств. Опубликован ряд обзоров и отчетов по различным аспектам диоксида титана, включая его свойства, получение, модификацию и применение: кратко рассмотрены важные особенности облученной поверхности \ce{TiO2} и представилен обзор типичных фотокаталитических реакций, наблюдаемых на гетерогенных дисперсных полупроводниках, также описаны эксперименты, которые помогают определить механизм такого фотокатализа [5]. Проанализированы некоторые принципы работы гетерогенного фотокатализа \ce{TiO2} [6].

\end{chapter}
	
\begin{chapter} {Методы получения наноструктурированных поверхностей \ce{TiO2}}
\section{Метод осаждения паров}
В последнее время широко исследуются методы осаждения из паровой фазы для изготовления различных наноматериалов, в том числе \ce{NS-TiO2} (NS - наноструктурированный - англ. nanostructured). В типичном процессе толстые кристаллические пленки \ce{TiO2} с размером зерна менее 30 нм, а также наночастицы \ce{TiO2} с размером менее 10 нм были получены путем пиролиза изопропоксида титана в смешанной атмосфере гелия/кислорода и осуществления доставки жидкого прекурсора. При осаждении на холодных участках реактора при температурах ниже $90^{\circ}$C с плазменным методом осаждения паров были получены и кристаллизованы аморфные наночастицы \ce{TiO2} с относительно высоким отношением площади поверхности к объему. Это происходит после отжига наночастиц при высоких температурах. Недостатками этого метода являются высокая температура процесса (около $1000^{\circ}$C), значительные размерные изменения и геометрические искажения изделий [4].

\section{Электрохимические методы}

Электрохимический метод часто используется для получения покрытия, обычно металлического, на поверхности путем восстановления на катоде. Особое внимание уделяется анодному окислению титана в различных электролитах. Влияние параметров синтеза, таких как плотность тока, концентрация электролита, приложенное напряжение и время анодного окисления, широко изучалось в [7]. Нанотрубоки диоксида титана можно получить на тонкой титановой фольге анодированием в \ce{HF}, содержащей водные растворы различных концентраций. Массивы нанотрубок постоянной длины с различным диаметром (25–65 нм) были получены при переменных напряжениях анодирования. Также обнаружено, что по мере увеличения напряжения наблюдается дисперсные или узловатые структуры, дискретно-полые цилиндрические трубки и губкообразная пористая структура [7]. Используя двухстороннее электрохимическое окисление титана в электролите, состоящем из воды, \ce{NH4F} и этиленгликоля, получают два высокоупорядоченных гексагональных плотноупакованных массива нанотрубок титана, разделенных тонким компактным оксидным слоем. Потенциостатическое анодирование титана в этиленгликолевом, \ce{NH4F} и водном электролите резко увеличивает скорость роста массива нанотрубок примерно до 15 мкм/ч, что представляет собой скорость роста примерно на 750–6000\% выше, чем наблюдаемая, соответственно, в других полярных органических соединениях. или электролиты на водной основе, ранее использовавшиеся для формирования массивов нанотрубок TiO2 [7].
\end{chapter}

% ============== Main content end ==============

% ============== Used literature start ==============
\newpage
%\addto\captionsrussian{\def\refname{Список используемой литературы}}
\addcontentsline{toc}{chapter}{Литература}
\addcontentsline{toc}{section}{Список литературы}
\begin{thebibliography}{}

    \bibitem{litlink1}  Juan M. Coronado • Fernando Fresno, María D. Hernández Alonso,
Raquel Portela --- Design of Advanced Photocatalytic Materials for Energy and Environmental Applications --- \it{Springer-Verlag London 2013} \normalfont

    \bibitem{litlink2}  Alireza Khataee, G Ali Mansoori  ---  Nanostructured Titanium Dioxide Materials --- \it{2012 by World Scientific Publishing Co. Pte. Ltd.
} \normalfont
    
    \bibitem{other-link-name}  А.А. Гончаров, А.Н. Добровольский, Е.Г. Костин, И.С. Петрик, Е.К. Фролова  ---  Оптические, структурные и фотокаталитические свойства наноразмерных пленок диоксида титана, осажденных в плазме магнетронного разряда --- \it {Журнал технической физики, 2014, том 84, вып. 6}. \normalfont
    
    \bibitem{other-link-name} Lai-Chang Zhang, Liang-Yu Chen, Liqiang Wang*  ---  Surface Modification of Titanium and Titanium Alloys: Technologies, Developments, and Future Interests --- \it {Adv. Eng. Mater. 2020, 22, 1901258}. \normalfont
    
    \bibitem{other-link-name} M.A. Fox, M.T. Dulay --- Heterogeneous photocatalysis --- \it{Chemical Reviews, 83, 341–357, (1993)}. \normalfont
    
    \bibitem{other-link-name} T. Yates, Jr, A.L. Linsebigler, G. Lu --- Photocatalysis on TiO2 surfaces: principles, mechanisms and selected results --- \it{Chemical Reviews, 95, 735–758, (1995)}. \normalfont
    
    \bibitem{other-link-name} H.E. Prakasam, K. Shankar, M. Paulose, O.K. arghese, C.A. Grimes --- A new benchmark for TiO2 nanotube array growth by anodization --- \it{Journal of Physical Chemistry C, 111(20), 7235–7241, (2007)}. \normalfont
    
\end{thebibliography}
% ============== Used literature end ==============

\end{document}

















