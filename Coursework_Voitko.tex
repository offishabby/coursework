\documentclass[oneside,final,12pt]{extreport}

\usepackage[T2A]{fontenc}
\usepackage[utf8]{inputenc}
\usepackage{vmargin}
\usepackage[english, russian]{babel}
%\usepackage{chemformula}
\usepackage[version=4]{mhchem}
\usepackage{indentfirst}

\begin{document}

% ============== Title page begin ==============

\centerline{БЕЛОРУССКИЙ  ГОСУДАРСТВЕННЫЙ УНИВЕРСИТЕТ}

\vfill
\vfill
\vfill
\large
\centerline{Войтко Глеб Георгиевич}
\vfill
\Large
\begin{center}
{\bf Наноструктурированные поверхности  диоксида титана для создания фотокаталитически активных гетероструктур}
\end{center}
\normalsize
\vfill
\centerline{Специальность 000.00.00 --- }
\centerline{Научно-исследовательская}
\vfill
\centerline{Курсовая работа}
\vfill
\vfill
\begin{flushright}
Научный руководитель:\\
д. Хорошко Л. С.
\end{flushright}
\vfill
\vfill
\centerline{Минск 2023}
\thispagestyle{empty}

% ============== Title page end ==============

% ============== Table of contents start ==============
\tableofcontents
% ============== Table of contents start ==============

% ============== Main content start ==============
\begin{chapter} {Введение}

Несомненно, возможность проведения таких сложных химических процессов, как инактивация бактерий или превращение \ce{CO2} в топливо, без использования высоких давлений или повышенных температур и с использованием света в качестве единственного источника энергии, интересна не только с инженерной точки зрения, но и с фундаментальной точки зрения. По этой причине в последние десятилетия активно проводятся исследования в области создания наноматериалов для фотокатализа. \\

В последние десятилетия исследования и разработки в области синтеза и применения различных наноструктурированных диоксидов титана (нанопроволоки, нанотрубки, нановолокна и наночастицы) приобрели огромный размах - в том чиле по причине наличия у подобных материалов фотокаталитических свойств
\end{chapter}
	
\begin{chapter} {Методы получения наноструктурированных поверхностей \ce{TiO2}}



\end{chapter}

% ============== Main content end ==============

% ============== Used literature start ==============
\newpage
%\addto\captionsrussian{\def\refname{Список используемой литературы}}
\addcontentsline{toc}{chapter}{Литература}
\addcontentsline{toc}{section}{Список литературы}
\begin{thebibliography}{}

    \bibitem{litlink1}  Juan M. Coronado • Fernando Fresno, María D. Hernández Alonso,
Raquel Portela --- Design of Advanced Photocatalytic Materials for Energy and Environmental Applications --- \it{Springer-Verlag London 2013} \normalfont

    \bibitem{litlink2}  Alireza Khataee, G Ali Mansoori  ---  Nanostructured Titanium Dioxide Materials --- \it{World Scientific Publishing Co. Pte. Ltd.} \normalfont
    
    \bibitem{other-link-name}  А.А. Гончаров, А.Н. Добровольский, Е.Г. Костин, И.С. Петрик, Е.К. Фролова  ---  Оптические, структурные и фотокаталитические свойства наноразмерных пленок диоксида титана, осажденных в плазме магнетронного разряда --- \it {Журнал технической физики, 2014, том 84, вып. 6} \normalfont
    
    \bibitem{other-link-name} Lai-Chang Zhang, Liang-Yu Chen, Liqiang Wang*  ---  Surface Modification of Titanium and Titanium Alloys: Technologies, Developments, and Future Interests --- \it {Adv. Eng. Mater. 2020, 22, 1901258} \normalfont
    
\end{thebibliography}
% ============== Used literature end ==============

\end{document}

















